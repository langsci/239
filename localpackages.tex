% add all extra packages you need to load to this file 
\usepackage{graphicx}
\usepackage{tabularx}
\usepackage{amsmath} 
\usepackage{multicol,multirow}
\usepackage{longtable}
\usepackage{lipsum}
%%%%%%%%%%%%%%%%%%%%%%%%%%%%%%%%%%%%%%%%%%%%%%%%%%%%
%%%                                              %%%
%%%           Examples                           %%%
%%%                                              %%%
%%%%%%%%%%%%%%%%%%%%%%%%%%%%%%%%%%%%%%%%%%%%%%%%%%%%
% remove the percentage signs in the following lines
% if your book makes use of linguistic examples
\usepackage{langsci/styles/langsci-optional} 
\usepackage{langsci/styles/langsci-lgr}
\usepackage[linguistics]{forest}

%% if you want the source line of examples to be in italics, uncomment the following line
% \def\exfont{\it}

\makeatletter
\let\pgfmathModX=\pgfmathMod@
\usepackage{pgfplots,pgfplotstable}%
\let\pgfmathMod@=\pgfmathModX
\makeatother
\usepgfplotslibrary{colormaps,colorbrewer}

\usepackage{siunitx}
\sisetup{output-decimal-marker={.},detect-weight=true, detect-family=true, detect-all, input-symbols={\%}, free-standing-units, input-open-uncertainty=, input-close-uncertainty= ,table-align-text-pre=false,table-align-text-post=false,uncertainty-separator={\,},group-digits=false,detect-inline-weight=math}
\DeclareSIUnit[number-unit-product={}]{\percent}{\%}
\makeatletter \def\new@fontshape{} \makeatother
\robustify\bfseries % For detect weight to work

\usepackage{langsci/styles/langsci-gb4e} 

\pgfplotsset{
colormap={langscicolormap}{color=(lsLightBlue) color=(lsRed)},
colormap name=langscicolormap
}
