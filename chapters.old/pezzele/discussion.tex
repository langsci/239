\section{Discussion}
\label{sec:discussion}

The present study investigated whether various, syntax-based classes of compounds (Subordinate, Attributive, Coordinate) can be described in terms of the quantitative, continuous properties of the meaning of the compounds and words belonging to them. To obtain these semantic measures, we generated cDSM representations for a list of compounds for which such classification was available. By running a series of logit models including both semantic and non-semantic factors as independent variables, we showed that our models are able to reliably capture different classes by means of semantic features.

\subsection{On the modifier-head similarity}

In particular, we showed that Coordinate compounds like \emph{comedy-drama} are predicted over either Subordinate (\emph{busstop}) or Attributive (\emph{halfprice}) by the higher semantic similarity between the head and the modifier. This finding is consistent with previous evidence from both theoretical linguistics and psychology. Within the lexical semantics approach, \cite{lieber5OHC} indeed proposed that Coordinate compounds are generated when the two constituents share almost identical `bodies' and `skeletons', that is, when the words to be combined have highly similar meanings.

Also, our finding is in line with several theories of conceptual combination, according to which Hybrid or Conjunctive interpretations would be produced by people for novel combinations which involve highly similar concepts, e.g. \emph{moose-elephant} (see among others \citealt{wisniewski1996}). Accordingly, and consistently with our results, Relation-linking interpretations (roughly equivalent to Subordinate compounds) would be instead produced for semantically highly dissimilar pairs, e.g. \emph{apartment-dog}. Since in our model the similarity between the constituents also distinguishes between Coordinate (Hybrid/Conjunctive) and Attributive compounds (Property-mapping), we propose that this result is consistent with the graded description proposed in many conceptual combination theories, where the difference between Property-mapping and Hybrid combinations would be due to an increasing number of both `commonalities' and `alignable differences' between the concepts to be combined \citep{wisniewski1996}.


\subsection{On the semantic role of compound constituents}

Second, we showed that Subordinate compounds are predicted against Attributive on the basis of the higher similarity between the compound and either constituents. That is, in compositionally-obtained Subordinate compounds both the modifier and the head contribute to a greater extent to the overall meaning than in Attributive ones. Moreover, the similarity between the compound and its head is a reliable predictor of Subordinate over both other classes.

First of all, these findings are again consistent with the lexical semantics literature \citep{sbg2005,lieber5OHC}. In it, Subordinate compounds are typically characterized by a structure in which the head \emph{selects} its argument. Therefore, the head contributes more to the overall meaning in this kind of compounds compared to the other classes, where a formal relation between the elements is absent. Also, these results are consistent with the different mechanisms proposed in the conceptual combination literature for Relation-based interpretations (capitalizing on a `slot-filling' procedure) and Property-based ones (where an `alignment' process is routinely carried out) \citep{wisniewskigentner,wisniewski1996}. In a nutshell, the slot-filling procedure would imply a bigger role of the compound \emph{head} compared to the other competing mechanism since, during combination, the head would be just \emph{filled} in one of its `slots' by the modifier concept.

Interestingly, these findings are also consistent with evidence from embodied cognition \citep{louwerse2008}. In particular, the embodied conceptual combination theory ({ECC}o) by \cite{lynott2009} proposes that the great majority of relational interpretations (corresponding to Subordinate compounds) are `non-destructive', namely, they result from the combination of constituent concepts that are left intact during the meshing of their `affordances'. To illustrate, in this approach the compound \emph{picture book} (i.e. `a book that has pictures’) is non-destructive, since the \emph{pictures} in question are still intact entities in the pages of the \emph{book}. Simplifying somewhat, the combinatorial procedure that leads to Relation-based interpretations (Subordinate) does not heavily modify the meaning of the original constituents, whereas Property-mapping ones (Attributive) are almost always destructive, that is, they involve the `destruction' of (part of) the constituent concepts. Using an example from \cite{lynott2009}, the compound \emph{icicle fingers} would reduce \emph{icicle} to a representation of `coldness' and `stiffness'. At the same time, the representation of the head (\emph{fingers}) would be switched toward a more figurative, metaphorical meaning, less similar to its prototypical representation (see also, e.g., \emph{iron curtain}). In this light, the similarity between either constituents as independent words and the compound will be generally higher in Relation-based (Subordinate) compared to Property-based interpretations (Attributive), given that the combinatorial procedure of the former type does not heavily modify the meaning of the original constituents. More in general, this observation provides indirect evidence that meaning representations extracted by texts via distributional semantics models can encode grounded information, at least to some extent \citep{louwerse2011}.

\subsection{On attributive compounds}

Third, compositionally-derived Attributive compounds are characterized by both a weaker contribution of the constituents in determining the overall meaning compared to Subordinate and a lower similarity between the constituents compared to Coordinate. This pattern of results is again consistent with \cite{lieber5OHC}, who proposes that Attributive formations emerge when the semantic
features of the constituents are too disparate to be interpreted in a Coordinative way and lack the argument structure that is typical of Subordinate compounds. Accordingly, Attributive compounds would represent a last-resort strategy used when the typical semantic features of the other classes are not satisfied \citep{lieber5OHC}. In line with such a description of Attributive compounds emerging \emph{in absentia} of discriminative features is evidence from conceptual combination showing that acceptability judgements for Property-based (Attributive) interpretations to novel compounds (e.g., a \emph{whale boat} is `a large boat’) are slower compared to Relation-based (Subordinate) interpretations (e.g. a \emph{whale boat} is `a boat for hunting whales') \citep{gagne2000}. According to \cite{gagne2015}, indeed, this would suggest that Relation-based interpretations are the product of an initial compositional process that, in the absence of the features that lead to either a relational interpretation (Subordinate) or a coordinate interpretation (Coordinate), leads to Property-mapping interpretations.

\subsection{On the methodology}

On the methodological level, it should be mentioned that we used a compositional model to generate representations for a list of compounds whose constituents were either nouns, verbs, adverbs, adjectives, etc. even though, in the training phase, only noun-noun compounds from CELEX were used. This could have represented a weakness for the system, causing the model to be biased toward noun-noun combinations. By looking at the results, however, we observed a similar, remarkably good performance of the model in all items, regardless of the grammatical category of the constituents. This is also clear by inspecting the examples in Table~\ref{tab:qualres}, where it can be noted that the parts-of-speech are almost uniformly distributed. However, it might be still possible that a richer training set might lead to even better results, perhaps achieving a better performance in generating meaning representations for less systematic, more opaque compounds. Indeed, we hypothesize that the lower accuracy obtained by the model opposing Attributive vs Subordinate compared to the others might be possibly due to this issue. Finally, we believe that the effectiveness of such an approach might be further validated by testing it on a larger (and possibly balanced with respect to compound type) set of annotated compounds. This, on the one hand, would strengthen the predictive power on the prediction task. On the other hand, it would allow more extensive, fine-grained analyses on the successes and failures of the models. We plan to further investigate this issue in future work.


\subsection{On the effectiveness of cDSMs in predicting compound relations}


The effectiveness of our approach in the proposed task is in line with previous work showing that compositional models of distributional semantics are successful in capturing relational information between the constituents of a compound. In particular, our task is related to that of predicting compound semantic interpretation (see section~\ref{sec:relnew}), where compositionally-obtained representations have been used to assign the correct semantic relation to noun-noun expressions. By experimenting with a number of cDSMs (including the one adopted in this study by~\citealt{guevara2010}), for example,~\cite{dima2016compositionality} obtained results comparable to state-of-the-art in 2 popular datasets~\citep{o2007annotating,tratz2010taxonomy}. Compared to SoA methods, however,~\cite{dima2016compositionality} only exploited information from word embeddings, thus proving the effectiveness of both distributed representations and compositional methods. In quantitative terms, our results are not directly comparable due to both the different experimental setting (we did not tackle the task as a classification problem) and the number of relations involved (3 vs either 6 or 43). Moreover, our results cannot be compared with previous work since, to our knowledge, we are the first in proposing this task. However, these studies jointly show that compositional  representations are successful in predicting compound relations defined on either \emph{semantic} or \emph{syntactic} bases.


\subsection{Final remarks}

In conclusion, this study suggests that different compound types identified on syntactic bases can be also defined in terms of continuous, quantitative features of the meaning of the compound and its constituents. We believe that discrete and continuous approaches are two faces of the same coin, the former representing a theoretically motivated, cross-linguistically valuable framework aimed at describing complex linguistic phenomena, the latter providing an interesting way to quantitatively test them. As indicated by our results, compositional models of distributional semantics represent a flexible and powerful way to capture many of these phenomena.

