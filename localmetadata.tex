\title{The role of constituents in multiword expressions}
\subtitle{An interdisciplinary, cross-lingual perspective}
% \BackTitle{Change your backtitle in localmetadata.tex} % Change if BackTitle != Title
\BackBody{Multiword expressions (MWEs), such as noun compounds (such
  as \textit{nickname} in English, and \textit{Ohrwurm} in German),
  complex verbs (such as \textit{give up} in English, and
  \textit{aufgeben} in German) and idioms (such as \textit{break the
  ice} in English, and \textit{das Eis brechen} in German), may be
  interpreted literally but often undergo meaning shifts with respect
  to their constituents. Theoretical, psycholinguistic as well as
  computational linguistic research remain puzzled by when and how
  MWEs receive literal vs.  meaning-shifted interpretations, what the
  contributions of the MWE constituents are to the degree of semantic
  transparency (i.e., meaning compositionality) of the MWE, and how
  literal vs. meaning-shifted MWEs are processed and computed.\\  
  This edited volume presents an interdisciplinary selection of six
  papers on recent findings across linguistic, psycholinguistic,
  corpus-based and computational research fields and perspectives,
  discussing the interaction of constituent properties and MWE
  meanings, and how MWE constituents contribute to the processing and
  representation of MWEs. The collection is based on a workshop
  %\footnote{\url{http://www.ims.uni-stuttgart.de/events/dgfs-mwe-17/}}
  at the 2017 annual conference of the German Linguistic Society
  (DGfS) that took place at Saarland University in Saarbrücken,
  Germany.}
%\dedication{Change dedication in localmetadata.tex}
\typesetter{Felix Kopecky}
%\proofreader{Change proofreaders in localmetadata.tex}
\author{Sabine Schulte im Walde\lastand Eva Smolka}
\BookDOI{10.5281/zenodo.3598577}%ask coordinator for DOI
\renewcommand{\lsISBNdigital}{978-3-96110-184-9}
\renewcommand{\lsISBNhardcover}{978-3-96110-185-6}
\renewcommand{\lsISBNsoftcover}{000-0-000000-00-0}
\renewcommand{\lsISBNsoftcoverus}{000-0-000000-00-0}
\renewcommand{\lsSeries}{pmwe} % use lowercase acronym, e.g. sidl, eotms, tgdi
\renewcommand{\lsSeriesNumber}{4} %will be assigned when the book enters the proofreading stage
\renewcommand{\lsID}{239} % contact the coordinator for the right number
