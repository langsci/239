\documentclass[output=paper]{langsci/langscibook} 

\title{The role of constituents in multiword expressions: An interdisciplinary, cross-lingual perspective}

\author{Sabine Schulte im Walde\affiliation{Institut für Maschinelle Sprachverarbeitung, Universität Stuttgart}\lastand  Eva Smolka\affiliation{Fachbereich Sprachwissenschaft, Universität Konstanz}}

% \chapterDOI{} %will be filled in at production

%\epigram{Change epigram in chapters/prefaceEd.tex or remove it there}

\abstract{\noabstract}

\maketitle

\begin{document}

\section{Introduction}

The processing and representation of multiword expressions (MWEs),
ranging from noun compounds (such as \textit{nickname} in English, and
\textit{Ohrwurm} in German) to complex verbs (such as \textit{give up}
in English, and \textit{aufgeben} in German) and idiomatic expressions
(such as \textit{break the ice} in English, and \textit{das Eis
  brechen} in German) has remained an unsettled issue over the past
20+ years.

Our research question concerns semantically transparent MWE
combinations as well as MWE combinations that result in a meaning
shift. For example, in the absence of situational experience, even
complex verbs that appear to be fully semantically transparent such as
\textit{aufstehen} (`stand up') do not necessarily have whole-word
meanings that are easily predictable from their constituents. Even
more difficult are complex verbs such as \textit{verstehen}
(`understand') and \textit{zustehen} (`legally due'), which contain
only a remote resemblance to the meaning of \textit{stehen}
(`stand'). Similarly, the constituents of noun compounds do not
necessarily contribute to their whole-word meanings in a
straightforward way. The meaning contribution may range from
relatively semantically transparent as in \textit{Nudelsuppe}
(`noodle soup') to semantically opaque, as in \textit{Spitzname}
(`nickname', lit. `pointy name'), \textit{Geduldsfaden} (`patience', lit. `patience thread'), or  \textit{Zwickmühle} (`dilemma',
lit. `pinch mill'), which contain a modifier (i.e. the left
constituent) and/or a head (i.e. the right constituent) that render the
compound semantically more opaque.  The most extreme meaning shifts
across types of MWEs occur in idiomatic constructions, such as
\textit{kick the bucket} and \textit{reach for the stars}, where the
literal meanings of the constituents do not seem to contribute to the
overall figurative meanings `die' and `strive for something
unachievable' at all. MWEs of the idiomatic type are typically assumed
to be semantically opaque, even though some idioms like \textit{spill
  the beans} are stronger in reflecting the figurative meaning
(`reveal a secret') in a metaphoric way than others.

This edited volume exploits complementary evidence across different
types of MWEs to shed light on the interaction of constituent
properties and meanings of MWEs. Specialists across languages and
across research disciplines contribute to this issue and provide a
cross-linguistic perspective integrating linguistic, psycholinguistic,
corpus-based and computational studies.


\section{Contributions}

In the following, the seven contributions in this volume discuss MWEs
that are composed of different types of constituents, including the
combination of particle+stem in complex verbs (e.g.,
\textit{aufstehen}, `stand up’), the combination of stem+stem in existing and novel compounds (e.g., \textit{nickname}, and \textit{campeel}, respectively), the combination of stem+preposition+stem in noun compounds (e.g., \textit{government assessment, juego de niños}), the combination of
modifier+stem in modifier-noun phrases (e.g., \textit{the brown dog})
and idiomatic combinations of words (e.g., \textit{reach for the
  stars}).

Sections~\ref{sec:verbs} to~\ref{sec:idioms} discuss the
interdisciplinary perspectives separately for complex verbs, noun
compounds and idiomatic expressions, and for each of these three
categories of MWEs we summarise the contributions to this collection.


\subsection{Complex verbs}
\label{sec:verbs}

Seminal psycholinguistic studies have applied manipulations of
semantic transparency to study whether verbal MWEs of the type
prefix+stem, particle+stem and stem+suffix are lexically represented
and processed via the constituents or as a whole-word unit (e.g.,
\citealt{Taft/Forster:75, MarslenWilsonEtAl:94, LongtinEtAl:03}).

Recurrent findings in English and French showed that semantically
transparent words facilitate their base (e.g.,
\textit{distrust--trust, confessor--confess}). This facilitation
result, however, was not obtained for semantically opaque primes
(e.g., \textit{retreat--treat, successor--success}). Lexicon-based
models concluded from these findings that a semantically transparent
word like \textit{confessor} possesses a lexical entry that
corresponds to its base and is represented as the stem
(\textit{--confess--}) and suffix (\textit{--or}), whereas
\textit{successor} is represented in its full form (e.g.,
\citealt{RastleEtAl:00, FeldmanEtAl:04, DiependaeleEtAl:05,
  Meunier/Longtin:07, MarslenWilsonEtAl:08, DiependaeleEtAl:09,
  Taft/NguyenHoan:10}).

Semantic transparency effects emerge also when transparency is
manipulated in a more graded way \citep{GonnermanEtAl:07}: Strong
facilitation effects showed for strongly phonologically and
semantically related word pairs (e.g., \textit{preheat--heat}),
intermediate effects for moderately similar pairs (e.g.,
\textit{midstream--stream}), and no priming for low semantically
related word pairs (\textit{rehearse--hearse}). Within learning-based
approaches, such as the convergence-of-codes account, form and meaning
relatedness between word pairs determines lexical processing
\citep{Plaut/Gonnerman:00, GonnermanEtAl:07}.

Findings in German, however, indicate that lexical processing occurs via the stem and irrespective of semantic transparency
(i.e., meaning composition of the complex verb). Low semantically
related word pairs (\textit{entwerfen--werfen}, `design'--`throw')
induced facilitation of the stem to the same extent as semantically
related word pairs did: \textit{bewerfen--werfen}, `throw at'--`throw'
(e.g., \citealt{SmolkaEtAl:09, SmolkaEtAl:14, SmolkaEtAl:15,
  SmolkaEtAl:18}). Most importantly, these findings stress the
importance of cross-language comparisons: what is true for the
processing in one language is not necessarily true for the processing
in another language \citep{GuentherEtAl:18}.

Computational approaches regarding the meanings of complex verbs have
mainly focused on predicting the degree of transparency of complex
verbs. These approaches typically rely on the distributional
hypothesis \citep{Harris:54,Firth:57} and empirical co-occurrence
information from large corpora, and are realised as vector space
models \citep{Turney/Pantel:10}. Regarding English, computational
approaches explored variants of distributional models and
distributional similarity, comparing word-based and syntax-based
descriptions, large-scale vs. dimensionality-reduced representations,
and verb-specific vs. general information
\citep[i.a.]{BaldwinEtAl:03,McCarthyEtAl:03,Bannard:05,Cook/Stevenson:06}.
Regarding German, an initial series of papers
\citep{Aldinger:04,SchulteImWalde:04a,SchulteImWalde:05,SchulteImWalde:06c}
studied particle verbs from a large-scale corpus-based perspective,
with an emphasis on salient distributional features at the
syntax-semantics interface. \cite{SchulteImWalde:06c} and
\cite{Bott/SchulteImWalde:18} integrated the subcategorisation
transfer of German particle verbs with respect to their base verbs
into models of
compositionality. \cite{Kuehner/SchulteImWalde:10}, \cite{Bott/SchulteImWalde:17},  and \cite{Koeper/SchulteImWalde:17a}
used clustering to distinguish between multiple senses, and common
cluster membership to determine
compositionality. \cite{Koeper/SchulteImWalde:16b} and
\cite{AedmaaEtAl:18} applied classifiers to identify figurative
language usage of German and Estonian particle verbs in context.

So far, most approaches that have dealt with complex verbs -- across
disciplines and across languages -- have considered semantic
transparency as the meaning relation between the whole word meaning of
the MWE and the meaning of its base constituent, disregarding the
contribution of the often ambiguous prefix or particle, e.g., they
were concerned with the question: to what degree is the meaning of
\textit{stand} reflected in \textit{understand}? Apart from a series
of formal word-syntactic analyses in the framework of
Discourse Representation Theory \citep{Kamp/Reyle:93} for
German particle verbs with the particles \textit{auf}
\citep{Lechler/Rossdeutscher:09}, \textit{ab} \citep{Kliche:11}. \textit{nach}
\citep{Haselbach:11} and 
\textit{an} \citep{Springorum:11}, this gap of knowledge has recently been
addressed from experimental perspectives: \cite{FrassinelliEtAl:17}
demonstrated in a lexical decision experiment that the particle
\textit{an} in German particle verbs is primarily associated with a
horizontal directionality, while \textit{auf} is primarily associated
with a vertical directionality. \citet{SchulteImWaldeEtAl:18} and
\citet{Koeper/SchulteImWalde:18} present data collections to assess
meaning components in German complex verbs. The former dataset
contains source- and target-domain characterisations of the base verbs
and the complex verbs, respectively, and a selection of arrows to add
spatial directional information to user-generated contexts; the latter
dataset contains ratings for strengths of particle-related pairs of
German base verbs and particle verbs.

As part of the present collection, \textbf{Springorum and Schulte im
  Walde} also focus on the meaning contribution of the particle to the
overall meaning of German particle verbs. They combine nine particles
(e.g., \textit{auf} `up') with 30 base verbs (e.g., \textit{geben}
`give') and examine how the particles are perceived in adding
directionality (e.g., up, down, left, right) to the meaning of the
particle verb (e.g., \textit{aufgeben} `give up'). That is, the
participants in their study saw a base verb or a particle verb and
decided which type of directionality in form of two-dimensional arrows
best reflects the verb meaning. Their qualitative and quantitative
analyses indicate that the particles exhibit individual spatial
profiles, but also that the particles vary in their flexibility to
provide predominant directions, in interaction with the abstractness
of the semantic base verb domains.


\subsection{Noun compounds}
\label{sec:compounds}

Compounds also lie on a continuum between relatively transparent and
rather opaque with respect to the meaning of their constituents.
Psycholinguistic research so far has been intrigued by the question
whether the compound is lexically represented and processed via the
constituents or as a whole-word unit. For example, findings on the
processing of noun-noun-compounds indicate a competition between the
compound’s constituents that correspond to independent words and their
whole-word counterparts. Hence, upon seeing the compound
\textit{doughnut}, the constituent [nut] may compete with the whole
word \textit{nut} (e.g., \citealt{Libben:06, FrissonEtAl:08,
  MonahanEtAl:08, Fiorentino/FundReznicek:09, Gagne/Spalding:09,
  Gagne/Spalding:14, Libben:14}). Another question concerns whether
the semantic transparency of the constituents affect the processing of
the MWE they compose, and if so, how? Indeed, semantically opaque
compounds are generally processed more slowly than semantically
transparent ones, and are less likely to show constituent activation –
probably because the semantic opacity of the whole compound makes its
constituents less relevant to lexical comprehension (e.g.,
\citealt{Taft/Forster:76, Sandra:94, Zwitserlood:94, IselEtAl:03,
  LibbenEtAl:03}). Furthermore, recent studies indicate that the
influence of semantic transparency is language-specific. The semantic
transparency of the head has been found to affect the processing of
noun-noun compounds in English and Italian (e.g.,
\citealt{MarelliEtAl:09, Marelli/Luzzatti:12}) but not in German (e.g.,
\citealt{Smolka/Libben:17}).

Computational approaches to predicting the transparency of noun
compounds can be subdivided into two subfields:
\begin{enumerate}
\item approaches that
aim to predict the \textit{meaning} of a compound by composite
functions, relying on the vectors of the constituents (e.g.,
\citealt{Mitchell/Lapata:10,CoeckeEtAl:11,BaroniEtAl:14b,Hermann:14}); and
\item approaches that aim to predict the \textit{degree of
  compositionality} of a compound, typically by comparing the compound
vectors with the constituent vectors (e.g.,
\citealt{ReddyEtAl:11a,Salehi/Cook:13,SchulteImWaldeEtAl:13,SalehiEtAl:14,SalehiEtAl:14b,SalehiEtAl:15,SchulteImWaldeEtAl:16b,Koeper/SchulteImWalde:17b}).

As for complex verbs, the computational models typically rely to a large
extent on the distributional hypothesis and empirical co-occurrence
information from large corpora. Individual research studies noticed
differences in the contributions of modifier and head constituents
towards the composite functions predicting compositionality
\citep{ReddyEtAl:11a,SchulteImWaldeEtAl:13}, but only a very limited
number of approaches zoomed into potentially relevant properties of
MWEs and their constituents, such as ambiguity, frequency and
productivity \citep{Bell/Schaefer:16,SchulteImWaldeEtAl:16b}.
\end{enumerate}

In this collection, \textbf{Pezzelle and Marelli} apply a
distributional semantic model to show that the semantic properties of
the compound and its constituents may explain syntactically-based
classes of compounds as suggested in linguistic theories
\citep{Bisetto/Scalise:05}. They differentiate between types of
compounds such as subordinate, attributive, and coordinate compounds,
on the basis of the underlying syntactic relation between the compound
constituents. In particular, Pezzele and Marelli provide measures that
quantify (a) the degree of semantic similarity between the
constituents, and (b) the contribution of each constituent to the
overall compound meaning, and show that these semantic measures are
effective in capturing the different syntactic linguistic classes. In
other words, the continuous quantitative semantic aspects of the
meanings of compounds parallel the discrete qualitative grammatical
distinctions between compounds.

\textbf{Iord\u{a}chioaia, van der Plas, and Jagfeld} study the
compositionality of English deverbal compounds. These deverbal nouns
are ambiguous between compositionally interpreted Argument Structure
Nominals, which inherit verbal structure and realise arguments
(e.g., \textit{assessment of the budget by the government}), and more
lexicalized Result Nominals, which preserve no verbal properties
or arguments (e.g., \textit{budget assessment}),
cf. \cite{Grimshaw:90}. While the former are fully compositional, the
latter remain ambiguous because the non-head (\textit{budget}) can be
interpreted as either subject or object. The authors apply
machine-learning techniques to evaluate corpus data and human
annotations to support their hypothesis and find that different
properties of the head contribute to the interpretation of the
deverbal compound.

In the third chapter on compounds, \textbf{Libben} investigates English compounds from a psycholinguistic perspective. He uses novel compounds such as \textit{anklecob} and \textit{clampeel}, the former being unambiguous, the latter being ambiguous in the way they can be parsed (i.e. \textit{ankle-cob} versus \textit{clam-peel} or \textit{clamp-eel}, respectively). A typing experiment shows that the typing latencies indeed peak at the morpheme boundary of nonambiguous compounds. Equivalent latencies at the critical letters of ambiguous compounds indicate that they are parsed in both possible reading ways. Libben refers to the heuristics of his Fuzzy Forward Lexical Activation account, which assumes that MWEs are parsed from left to right for any possible word combination. He concludes that complex words are not static representations but rather patterns of actions.

Two papers deal with MWEs that are untypical compound constructions
for which linguistic theories in general refer to the notions of
lexicon and syntax and debate whether these MWEs are to be considered
as compounds or not. \textbf{Hennecke} examines the formation of MWEs
of the type ``N Prep N'' in Romance languages, such as Spanish, French
and Portuguese (e.g. \textit{juego de niños}, `kid’s game') and takes
a constructionist approach to analyse the constructions as abstract
templates. In a qualitative analysis, she examines the variation that
the preposition in a construction may undergo (e.g. \textit{juego de
  niños} vs. \textit{juego para niños}, both meaning `kid’s game'). To
this end, she analyses the semantic relations between the nominal
constituents and the semantic transparency of the constructions. Her
findings indicate that variability of the prepositional element occurs
only in semantically transparent constructions. Furthermore,
prepositional variability largely varies across the three Romance
languages.

Also \textbf{Gagné, Spalding, Burry, and Adams} examine MWEs that are
not typically classified as compounds and compare modifier-noun
phrases (e.g., \textit{the brown dog}) with full phrases (e.g.,
\textit{the dog that was brown}). They examine how modifying
information that refers to recently encountered information is used in
the production of MWEs and manipulate the property of the head noun
between normal (e.g., \textit{brown}) and distinctive (e.g.,
\textit{blue}). Participants showed a strong overall bias toward using
a modifier-noun phrase structure (regardless of whether they
previously saw a modifier-noun phrase or a full phrase), and were more
likely to include distinctive properties (\textit{the blue dog}) than
normal properties (\textit{the brown dog}) when referring to the
concept. These findings indicate that modifier-noun phrases have a
privileged status among MWEs and provide a good compromise between
conveying sufficient information and using simple syntactic
structures.


\subsection{Idioms}
\label{sec:idioms}

Idiomatic expressions are the MWEs which may be considered as showing
the strongest semantic shift that the constituents undergo, because
the figurative meaning is usually not even remotely connected with the
meaning of its constituents, as in \textit{hit the road}. Rather,
idiomatic expressions are considered semantically fixed, since the
figurative meaning does not allow the replacement of any of the word
constituents (e.g., \textit{*she hit the street; *she beat the road}),
and the modification of an idiomatic constituent is assumed to change
the figurative meaning into a literal meaning.

The processing and representation of idioms has thus remained an
unsettled issue in psycholinguistic research: how is the figurative
meaning processed and stored in lexical memory? In particular, is the
figurative meaning of an idiom represented separately from the meaning
of its constituents, and how is the figurative meaning assembled
(e.g., \citealt{Cacciari/Tabossi:88, Gibbs:92, Cacciari/Glucksberg:94,
  Titone/Connine:99, Hamblin/Gibbs:03})? Seminal studies thus assumed
a ``non-compositional'' representation in which the whole figurative
meaning of an idiom is stored as a distinct entry in the mental
lexicon similar to the representation of a complex word like
\textit{Finanzmarktaufsichtsbehörde} `financial market supervisory
authority' (e.g., \citealt{Bobrow/Bell:73, Swinney/Cutler:79,
  Gibbs:80}). More recent hybrid models try to integrate the
assumption that idioms are both compositional and unitary: On the one
hand, an idiom is composed of single constituents that are
activated to some degree, and on the other hand each idiom possesses
its own lexical entry that stores the whole meaning of the idiom
(e.g., \citealt{Cacciari/Tabossi:88, GibbsEtAl:89, Cutting/Bock:97,
  Titone/Connine:99, SprengerEtAl:06, Caillies/Butcher:07,
  Holsinger/Kaiser:13, Titone/Libben:14}).

As far as computational work on idiomatic expressions is concerned,
several research studies measured the syntactic flexibility of
idiomatic expressions, to a large extent focusing on verb--object
combinations (e.g., \citealt{Bannard:07,FazlyEtAl:09}). These measures
varied the constituents of the target MWEs, explored modifiability and
passivisation, etc. in order to distinguish between literal
vs. idiomatic interpretations. A large number of automatic
classification approaches addressed idioms as non-literal language
identification across various types of MWEs, mostly relying on
contextual indicators to distinguish between literal and idiomatic
interpretations (e.g.,
\citealt{Sporleder/Li:09,TurneyEtAl:11,Koeper/SchulteImWalde:16b}),
such as distributional similarity, text cohesion graphs, and
contextual abstractness. The variation-based approaches further
provide some insight into the flexibility of MWE constituents and
their meaning contributions.

The last paper by \textbf{Smolka and Eulitz} deals with idioms and how
the meaning of the constituents contributes to the figurative meaning. They
present three experiments, in which participants rate the meaning
similarity between an idiomatic phrase (e.g., \textit{She always
  reached for the stars}) and a paraphrase of its figurative meaning
(e.g., \textit{She always strove for something unreachable}). They
exchange the noun, verb, or prepositional idiomatic constituent by a
close semantic associate (e.g., \textit{She always reached/grasped
  for/at the stars/planets}) and find that a modified constituent
still preserves the figurative meaning. This study adds to the
understanding that there is no completely fixed unitary entry and
that the idiomatic constituents do contribute to the figurative
meaning of the idiom, even though the figurative meaning is
semantically opaque.


\vspace{+5mm}
\section*{Acknowledgements}

This collection was supported by the DFG Collaborative Research Centre
SFB 732 and the DFG Heisenberg Fellowship SCHU-2580/1 (Sabine Schulte
im Walde), and by the Volkswagen Foundation Grant FP 561/11 (Eva
Smolka). Special thanks go to our student researcher Anurag Nigam who
type-set this volume.
%
Last but not least we thank our experts from the interdisciplinary
fields who ensured a qualitatively high-standing reviewing process:
% % \vspace{-1mm}
\begin{itemize}
% %  \itemsep -1mm
%\item Timothy Baldwin (University of Melbourne, Australia)
\item Melanie Bell (Anglia Ruskin University, UK)
\item Jens Bölte (University of Münster, Germany)
%\item Fabienne Cap (Uppsala University, Sweden)
\item Paul Cook (University of New Brunswick, Canada)
\item Christina Gagné (University of Alberta, Canada)
\item Giannina Iord\u{a}chioaia (University of Stuttgart, Germany)
\item Alessandro Lenci (University of Pisa, Italy)
\item Marco Marelli (University of Milano-Bicocca, Italy)
\item Carlos Ramisch (Aix-Marseilles University, France)
\item Martin Schäfer (University of Jena, Germany)
\item Nils Schiller (Leiden University, The Netherlands)
\item Thomas Spalding (University of Alberta, Canada)
\item Lonneke van der Plas (University of Malta, Malta)
\item Aline Villavicencio (Universidade Federal do Rio Grande do Sul, Brazil / University of Essex, UK)
\end{itemize}


% \clearpage
% Bibliography:

\vspace{+5mm}
{\sloppy \printbibliography[heading=subbibliography,notkeyword=this]}


\end{document}
