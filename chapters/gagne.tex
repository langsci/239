\documentclass[output=paper]{langsci/langscibook} 

\title{Production of multiword referential phrases: Inclusion of
  over-specifying information and a preference for modifier-noun
  phrases}

\shorttitlerunninghead{Production of multiword referential phrases}
  
\author{Christina L. Gagné \affiliation{Department of Psychology, University of Alberta}\and  Thomas L. Spalding \affiliation{Department of Psychology, University of Alberta}\and  J. Claire Burry \affiliation{Department of Psychology, University of Alberta}\lastand  Jessica Tellis Adams\affiliation{KidsAbility Centre for Child Development, Ontario}}

% \chapterDOI{} %will be filled in at production

% \epigram{}

\abstract{We examined the underlying psycholinguistic and cognitive factors that give rise to the production of multiword expressions. For example, if a story describes a woman buying a dog with blue fur, will people include the color of the dog when referring to the animal and, if so, in what syntactic form? In the experiment, participants read short stories that contained a concept that was presented as either a modifier-noun phrase (e.g., the blue dog) or full phrase (e.g., the dog that was blue). We also varied whether the property being highlighted was normal (e.g., brown) or distinctive (e.g., blue) for the head noun concept (e.g., dog). We found that participants are more likely to include distinctive properties than normal properties when referring to the concept. Although the selection of a syntactic form was partially influenced by the form of the information in the story, there was a strong overall bias toward using a modifier-noun phrase structure.}


\begin{document}
\maketitle


\section{Introduction}

\subsection{Aim and background}

The aim of this chapter is to explore when and how multiword
expressions are used within a referential context. In particular, we
focus on production of referential expressions and examine what drives
the inclusion of modifying information and the syntactic form of the
expression. When referring to an object, person, or event, a
speaker/writer is faced with the challenge of assigning linguistic
labels to conceptual entities; often, several linguistic expressions
can be used. For example, the same object can be referred to as “cup”,
“ceramic cup”, or “cup that is made of ceramic”. What influences this
decision? Two aspects of forming a referential expression are
particularly relevant and will be the focus of our
investigation. First, the speaker/writer might or might not include
modifying information in the referential expression. Second, if
modifying information is included, the expression might be a compound
(e.g., ceramic cup) or a full noun phrase (e.g., cup that is made of
ceramic). Although it is tempting to think of these as two separate
ordered decisions (first decide whether or not to modify, then decide
the form of the modification) we should note that these two aspects
are not necessarily deliberate, conscious choices, nor need they be,
strictly speaking, independent or sequential. Rather, the ultimate
form of the expression may reflect underlying cognitive processes
carried out within the language system that, working together, give
rise to the form of the expression, and hence to both the syntax and
the presence (or not) of modifying information.

Much of the existing work on compounds and modifier-noun phrases has
focused on compound access and interpretation. The current study takes
a different approach to this problem. Rather than focusing on the
interpretation per se, we examine production to identify some of the
expectations and biases that human users have about the use of
modifying information during referential communication. When using
referential expressions, speakers/writers attempt to establish both
semantic co-ordination and lexical co-ordination with the addressee
(e.g.,
\citealt{clark1986referring,garrod1987saying,clark1989contributing}). An
attempt is made to synchronize the underlying mental model of the
current situation as well as the specific expressions that are applied
to particular entities within that model. In doing so, the
speaker/writer draws on many different types of knowledge, including
world knowledge, knowledge about information expressed in the
conversation/discourse, and knowledge about linguistic
conventions. Identifying the expectations that people have about the
use of multiword expressions provides insight into how people are
conceptualizing both the entities denoted by these constructions and
the scenarios in which the constructions are or should be used.
Consequently, this area of research has implications for a variety of
areas within the psycholinguistic and linguistic literature. In
particular, the current project contributes to research that examines
the contribution of the individual constituents to the understanding
of the meaning of the whole expression, and the appropriateness of the
use of the whole construction in a given situation.

The semantic transparency of the constituents of a compound has been a
widely studied aspect of compound processing \citep{libben1998semantic,jarema1999processing,gagne2016effects,smolka2017can}. In general,
compounds with opaque constituents (e.g., \textit{humbug}) are more
difficult to process than compounds with transparent constituents
(e.g., \textit{schoolyard}). Of course, in creating a multiword
referential phrase that is new (as opposed to a known compound word,
for example), the constituents will need to be relatively transparent
in order to provide the information that would allow the communicative
task to be successfully completed. However, at any given level of
transparency there are other aspects that will influence whether a
head noun is modified. In the current chapter, we will consider one of
these factors, namely the distinctiveness of the property denoted by
the modifier, which, like semantic transparency, is a semantic factor.
Both \textit{blue dog} and \textit{brown dog} are semantically
transparent expressions in that the meaning of the constituents
contribute to the meaning of the whole. However, \textit{blue dogs}
are more distinctive compared to the concept \textit{dog} than are
\textit{brown dogs}. We explore whether people are sensitive to the
distinctiveness of a property during the formation of multiword
expressions.

\subsection{Overview of chapter}

In this chapter, we begin by providing an overview of the theoretical
issues concerning the inclusion of modifying information and the use
of either full phrases or modifier-noun phrases. Next, we present an
experiment in which we manipulated two factors that might influence
the production of referential expressions. In particular, we examined
whether the distinctiveness of the modifying information influences
whether that information is used when referring to the antecedent. In
addition, we examined whether the syntactic form in which the
modifying information is presented influences the form in which
modifying information is conveyed. Finally, we discuss the relevance
of the empirical data within a psycholinguistic context and highlight
the implications of the data for multiword expressions and for
modifier-noun phrases in particular.

\subsection{What motivates the inclusion of modifying information?}

The expressions used to denote referents reflect how the
speaker/writer is conceptualizing the object and, in particular, how
he/she chooses to distinguish it from other items \citep{brown1958shall,olson1970language}. Indeed, speakers are sensitive to both nonlinguistic- and
linguistic-ambiguity during referential communication and attempt to
avoid producing ambiguous expressions \citep{ferreira2005speakers}. A key issue for the current research concerns the factors that
lead people to include modifying information rather than using an
unmodified noun when producing a referential expression. The inclusion
of modifying information serves several linguistic and psychological
functions. Most often, modifying information is used to distinguish
among potential referents \citep{downing1977creation,brekle1986production}. There are often
situations in which using the category label alone would not be
sufficient. Consider a situation in which there are several cups on a
table. To refer to a particular cup, for example, a speaker might
specify its material and use either a full noun phrase (e.g., May I
have the cup that is ceramic?) or a compound (e.g., May I have the
ceramic cup?). Both utterances involve combining information about the
head noun concept (e.g., cup) with information about a modifying
concept (e.g., ceramic). This combination of information, in turn,
allows the unambiguous identification of the referent within the
available set of potential referents.

Several experiments on referential communication that used a visual
display of objects (\citealt{tanenhaus1995integration}; see also
\citealt{frank2012predicting}) have found that speakers use a
pre-nominal adjective (e.g., tall glass) in a context in which there
are contrasting members (e.g., a short glass), which is consistent
with the hypothesis that speakers try to make their utterances as
informative yet as economical as possible \citep{grice1975logic}. The
pre-nominal adjective is used to uniquely identify one object among
several objects. However, the motivation for using modifying
information appears to go beyond merely disambiguating among multiple
possible referents because it is often included even when there is no
need to provide additional information. This phenomenon of providing
modifying information even in cases where such information is not
needed to identify the referent is known as
over-specification. Indeed, there are a number of studies showing that
participants include adjectives during referential communication even
though this additional specification is not required to identify the
referent
(e.g. \citealt{pechmann1989incremental,sedivy2003pragmatic,maes2004reference,koolen2013effect}).

Over-specification performs various functions in addition to
identifying referents. For example, modifying information (e.g., the
cup that is on the shelf near the plate) is used to shift the
addressee’s focus of attention \citep{mira1990accessing,prince1992zpg,gundel1993cognitive,chafe1994discourse}. Another reason for using
modifying information is to conform to pre-established conversational
pacts \citep{brennan1996conceptual,ibarra2016flexibility}. Conversational partners often converge on an expression and
will persist in using that expression even when there is no longer a
need to include the additional information. To use an example from
\citet{brennan1996conceptual}, the term \textit{pennyloafer} was initially
used to denote a particular shoe among other possible shoes. However,
the speaker continued to use this term rather than switching to using
the simpler term \textit{shoe} even when no other shoes were present
in the display.

From a cognitive processing perspective, over-specification appears
beneficial to both the speaker and the listener. For example, it aids
in the identification of objects in a visual array and, consequently,
speakers are more likely to produce over-specified expressions when
they were asked to imagine that the task was very important (i.e.,
when told to imagine that the control panel is being used for
long-distance surgery) than when they were not given such a scenario \citep{arts2011}. Over-specification also benefits production \citep{pechmann1989incremental}. Consistent with this idea, redundant information is
more likely to be included when the speaker is under time
pressure. \citet{koolen2016distractor} conducted a study in
which participants referred to target objects in a visual array of
objects. Participants were more likely to use over-specifying
information when they were under a time constraint (e.g., they had to
respond within 1000 ms) than when they had an unlimited amount of time
to refer to the target object. \citet{koolen2016distractor} concluded that
when individuals are under pressure, they are more likely to use quick
heuristics and therefore select properties of an object based on their
perceptual salience rather than discriminatory power.

Overall, there appear to be many reasons for why speakers might choose
to include modifying information in referential expressions. In the
current experiment, we focus on additional usage of modifying
information that has not been fully explored in the literature. In
particular, we propose that modifying information might be used to
mark a conceptual distinction among category members and, in
particular, to make explicit note of particularly distinctive
information.

Studies on referential expressions within a visual context (i.e.,
situations in which objects are presented visually) indicate that the
distinctiveness of visual properties within the display influences
referential expressions. Participants were more likely to provide
modifying information (i.e., to produce over-specified expressions)
when the property of an object is atypical (e.g., \citealt{westerbeek2015stored}). For instance, \citet{rubio2016redundant} used a referential
communication task in which participants asked the researcher to click
on objects that were presented in an array on the computer screen. In
the first experiment, participants saw pictures of paper dolls and a
display of paper clothes that were either all the same color (e.g.,
brown purse, brown shirt, brown dress, and brown shoes) or different
colors (e.g., yellow purse, pink shoes, blue dress, and red pants). In
the second experiment, participants saw arrays of animals, fruits, vegetables,
and artifacts that either had typical colors (e.g., brown camel) or
atypical colors (e.g., blue camel).  Participants tended to use a
redundant color adjective in instances where such modifying
information would be unnecessary (e.g., \textit{“the blue dress”},
where only one dress could be a possible referent) more often when the
object was an atypical color than when the object was a stereotypical
color. These results suggest that modifying information is used when
the concept has been modified with a distinctive
property. Furthermore, participants provided modifying information
more often when the color was a central property of the object
category (e.g., referents such as clothing yielded a higher usage of
redundant color adjectives than did geometrical figures). Taken
together, these results suggest that a key characteristic in terms of
determining whether modifying information is provided is conceptual
distinctiveness rather than perceptual/visual distinctiveness.  That
is, the distinctiveness of the information relative to the category
itself, rather than just within the visual display.

The aim of the current study is to explore the role of conceptual
distinctiveness by examining whether the tendency to mention
distinctive properties extends to situations in which the objects are
not physically present. In particular, we will focus on a situation in
which the contrast with other category members is implied or based on
conceptual knowledge within a story context, rather than presenting
the objects in a visual display. For example, mentioning that flowers
are either fresh or wilted implicitly contrasts the flowers with ones
that are not fresh or not wilted. Moreover, in the context of buying
flowers as a gift, it is more typical to buy ones that are fresh than
ones that are wilted. Thus, from a conceptual perspective, the
property \textit{wilted} is more distinctive for flowers than is
\textit{fresh}.

Conceptual distinctiveness is related to the issue of contrast. The
notion of contrast between categories and subcategories has long
played an important role in linguistic and psycholinguistic
theories. Indeed, the principles of contrast and mutual exclusivity \citep{clark1983lexicon,carstairs2010evolution} are well-known constraints on
word learning. In terms of multiword expressions, previous research on
conceptual combination suggests that the notion of contrast influences
how people use noun phrases. For example, \citet{gagne1996influence}
found that when verifying whether a property is true of a
modifier-phrase (e.g., \textit{submarine door}), people took less time
to verify a property that was true of the phrase but not generally
true of the head noun (e.g., \textit{made of metal}) than to verify a
property that was true of both the phrase and the head noun (e.g.,
\textit{solid}). This finding suggests that people are sensitive to
the extent to which the modified concept (e.g., \textit{submarine
  door}) is semantically/conceptually distinctive from other members
of the head noun concept (e.g., \textit{door}).

In terms of judgments about whether a concept has a particular
property, several studies \citep{connolly2007stereotypes,gagne2011inferential,hampton2011modifier,jonsson2012modifier,gagne2014subcategorisation} have shown that properties that are true of
the head noun (e.g., \textit{kites have strings}) are viewed as being
less true of the modified head (e.g., \textit{silk kites have
  strings}). This effect (known as the modification effect) appears to
be driven by the expected level of contrast between the combined
concept (e.g., \textit{silk kites}) and the head concept (e.g.,
\textit{kites}); when making judgments about the likelihood that a
property is true, participants are influenced by the meta-knowledge
that modified concepts are used to signal that the subcategory is
similar to the category (e.g., silk kites have many properties in
common with kites) but also that the subcategory is somehow different
than the category \citep{gagne2011inferential,gagne2014subcategorisation,spalding2015property}. These two expectations account for why
properties that are true of the head noun are judged as being less
true of the modified concept, and that properties that are false of
the head noun (e.g., \textit{candles have teeth}) are judged to be
more true (but still unlikely) of the modified concept (e.g.,
\textit{purple candles have teeth}). Indeed, the effects of the
expected contrast is so strong that the same effects are seen even
when the modifier is a non-word (e.g., \citealt{gagne2015semantics}).

Thus, we conclude that conceptual contrast or conceptual
distinctiveness is a critical factor in the use and understanding of
multiword phrases and compound words in general and is therefore
likely to contribute to the production of such phrases.

\subsection{When modifying information is included, how is it expressed?}

If modifying information is included, the syntactic form which
expresses this information can still vary. In English, modifying
information can be expressed as a full noun phrase (e.g., a dog that
is blue) or as a modifier-noun phrase (e.g., a blue dog). Do people
have a priori biases toward using one linguistic expression over
another? The answer is not immediately obvious because intuitions
based on ease of processing do not correspond with the tendency for
expressions to become shortened over time.

In terms of ease of processing, there is an advantage to using a full
phrase because noun compounds are particularly challenging to
interpret
\citep{lapata2002disambiguation,copestake2005noun,libben2014nature}. Much
of the difficulty lies in recovery of an implicit underlying relation
between the modifier and head noun concept. A modifier-noun phrase is
more ambiguous than a full phrase, in that the full noun phrase
explicitly describes the exact nature of the modification that is
being performed (e.g., \textit{oil for babies}) whereas, for
modifier-noun phrases (e.g., \textit{baby oil}) the nature of the
modification is implicit and must be reconstructed by the
listener/reader (see \citealt{levi1978syntax,gagne1997influence}). The
term “modifier-noun phrase” most often refers to constructions that
are novel (e.g., \textit{apple juice seat}; \textit{mountain
  magazine}), but, can also refer to lexicalized open (unspaced)
compounds (e.g., \textit{hunting dog}; \textit{paper bag}). Indeed
there seems to be commonalities in the processing of novel noun
phrases and lexicalized compounds \citep{gagne2006using}. Psycholinguistic research has shown that human language users actively make use of relations during the processing of
both novel and established/lexicalized compounds
\citep{gagne1997influence,gagne2002lexical,gagne2009constituent,gagne2014conceptual}. This
research indicates that, during the comprehension of noun compounds,
the more available the required relation is, the easier it is to
select the relation and, consequently, the less time it takes to
interpret the compound. In other words, the more difficult it is to
recover the implicit underlying relation, the more difficult it is to
interpret a compound (see, for example,
\citealt{gagne1997influence,spalding2014relational,schmidtke2018conceptual}).

Given the difficulty inherent in recovering implicit semantic
relations, one would presume that it would be advantageous to overtly
express the relation and, consequently, to avoid the use of
compounds. Yet, this is not what happens within the human language
system. Over time, lexicalized phrases are often truncated and become
compounds (e.g., \textit{our lady’s bug} became
\textit{ladybug}). Similarly, compounds can become non-compounds
(e.g., \textit{electronic mail} became \textit{e-mail} and, more
recently, \textit{email}); the words \textit{lord} and \textit{lady}
are derived from Old English compounds \textit{half-weard}
“bread-keeper” and \textit{halfdige} “bread-kneader”. This truncation
that occurs on a global (and more long-term) level within a language
also occurs during local interactions.  During referential
communication, for example, linguistic expressions are often shortened
\citep{garrod1987saying,brennan1996conceptual}. For example, in one
experiment, a geometric figure that was initially described as looking
“\ldots like a person who’s ice skating, except they’re sticking two
arms out in front” became “the ice skater”
\citep{clark1986referring}. Similarly, an object that was initially
referred to as “the car that has like \ldots blueprints painted on the
side of it sorta” was later referred to as “the blueprint car”
\citep{metzing2003conceptual}. In sum, there appears to be a
preference toward using syntactically simpler expressions such as
compounds, even though such expressions are inherently more ambiguous
than full expressions which specify the relation overtly.

On the basis of these findings, one would expect an overall bias
towards using a truncated expression (e.g., using \textit{wilted
  flowers} or \textit{even flowers}, rather than \textit{flowers that
  are wilted}).  However, this bias must also be considered in light
of another bias reported in the literature -- namely, the tendency for
people to re-use recently encountered syntactic structures. For
example, \citet{bock1986syntactic} demonstrated that speakers tend to re-use a
syntactic structure from the priming sentence when describing a
scene. This effect has been examined in a variety of context including
examinations of whether it can be driven by a single word as in the
case of featural accounts of syntactic priming. For example, \citet{melinger2005lexically} found that production preferences for dative
alternation can be biased by prior exposure to a single verb. However,
most relevant for the current project concerns studies that focus on
the creation of referential expressions. Syntactic convergence occurs
during referential communication. For example, participants were more
likely to describe a picture of a red sheep as “The sheep that’s red”
when the confederate recently described a picture of a red door as
“The door that’s red” than when it was described as “a red door” \citep{cleland2003use}. This result suggests that participants
tend to re-use syntactic structures, especially when the prime and
target sentences share lexical items such as \textit{red} (see also
\citealt{chang2003can}). Similarly, \citet{tarenskeen2015overspecification} found that when participants use modifying information
to describe a target item from a visual array of six drawings of
clothing, there is a tendency to continue to re-use the same syntactic
structures.

These studies all demonstrate that participants have a tendency to
re-apply the same syntactic structure that was used with one
object/entity (e.g., \textit{sheep}) when subsequently referring to a
separate object/entity (e.g., \textit{door}). However, an unresolved
question concerns whether syntactic priming will occur in a task in
which participants are introduced to a concept (e.g., \textit{apples
  that are rotten}) and then are asked a question requiring them to
refer to that same concept.  This situation directly pits the bias
towards truncation against the bias towards re-using syntactic
expressions. The current experiment will investigate this issue.


\section{Experiment}

We examine the types of referring expressions that people produce when
referring to a concept that has been encountered in a short
description of a scenario. The experiment was designed to address two
key issues: 1) whether the distinctiveness of the modifying
information being conveyed about a target entity in a story influences
whether that information is included when the participant is asked to
refer back to the entity and 2) whether the syntactic form in which
the modifying information is presented influences the form in which
modifying information is conveyed. Participants read short stories and
then answered a comprehension question that would require them to
refer to something in the story. For example, one story described a
woman buying a pet. The target antecedent was the dog that she
purchased. We varied the type of modifying information that was
presented with the target antecedent. The information was either
normal or typical for the object or was distinctive. To illustrate,
all participants read a version of the story in which the color of the
dog was mentioned. For half of the participants, the dog was described
as having brown fur (a normal feature for dogs), and for the other
half, the dog was described as having blue fur (a distinctive feature
for dogs). We were interested in what the participants would produce
when they were asked \textit{What kind of pet did Sally buy?}

We predict that distinctiveness will influence whether participants
choose to include modifying information in their linguistic
expression. Properties that are unusual or distinctive for the head
noun will be seen as especially relevant and, consequently, will be
more likely to be included in the description provided by the
participants. However, properties that are not unusual will be deemed
less relevant (because the majority of members of the head noun
category have the same property) and therefore less likely to be
included. Thus, when referring to a dog that was previously mentioned
in a short story, participants will be more likely to include
modifying information when the dog was described as having an atypical
color such as \textit{blue} relative to when the dog was described as
having a typical color such as \textit{brown}, because the resulting
subcategory is more distinctive and therefore will tend to more
readily identify the appropriate referent. In short, there are lots of
brown dogs, but relatively few blue dogs in the world, and,
consequently, it should be more informative to refer to the
subcategory of \textit{blue dogs} than to the subcategory of
\textit{brown dogs}. Note, however, that in no case is the modifying
information required to uniquely identify the referent.

In terms of the syntactic form that is used to convey the modifying
information, the existing literature points to two conflicting
predictions. On one hand, people might show a tendency toward using a
modifier-noun phrase even when the information is presented as a full
noun phrase. Two considerations arise here. First, the modifier-noun
phrase is shorter and syntactically simpler and, thus, might generally
be preferred. Second, a modifier-noun phrase is more ambiguous than a
full phrase, in that the full noun phrase explicitly describes the
exact nature of the modification that is being performed (e.g., a dog
that is blue) whereas, for modifier-noun phrases the nature of the
modification is implicit and must be reconstructed by the
listener/reader \citep{downing1977creation,levi1978syntax}. Having the
relation directly specified (e.g., \textit{crayon that is made of
  plastic}, or \textit{sunshine in the morning}) removes uncertainty
about relation selection
\citep{gagne2014conceptual,gagne2015semantics}. Thus, there could be
some trade-off, in which speakers or writers generally prefer to use
the shorter modifier-noun phrase, as long as they have reason to
believe that the recipient will understand the implied connection
between the modifier and the head noun
concepts. \citet{gagne2004effect} found that the presence of a
referent in a discourse made modifier-noun phrases easier to
comprehend, even though the phrase itself had not been presented. In
the present study, all of the stories include information (either in
the form of the full noun phrase or the modifier-noun phrase) that
should make it easy for a recipient to understand the modifier-noun
phrase. Therefore, the participants, in responding to the question
about the target antecedent, might show a general preference for the
modifier-noun phrase.

On the other hand, the form in which the information was initially
presented in the preceding discourse might influence the manner in
which the information is later conveyed due to syntactic priming. That
is, when information is presented as a modifier-noun phrase, then
people should be more likely to produce a modifier-noun phrase than
when the information is presented as a full noun phrase. This
prediction is derived from research on the activation of syntactic
structure during speech production that demonstrates that speakers
tended to reuse a syntactic structure from the priming sentence when
describing a scene \citep{bock1986syntactic,bock1990framing}.

\subsection{Method}

\subsubsection{Participants}

Fifty-four introductory psychology students participated for partial
course credit. All participants were native speakers of English. The
data from two participants were not used because they did not follow
instructions. Thus, data from 52 participants were included in the
analyses.

\subsubsection{Materials and procedure}

Twenty-eight short stories were constructed. Each story was under 65
words long and contained a target antecedent (i.e., the antecedent
that we will be eliciting) for which we provided modifying
information. We varied whether the modifying information was
distinctive (e.g., \textit{blue fur}) or usual (e.g., \textit{brown
  fur}) for the head noun (e.g., \textit{dog}) in the context of the
story. In addition, we varied the syntactic form in which the
modifying information was presented: the information was presented as
a modifier-noun phrase (e.g., \textit{brown dog; blue dog}) or full
noun phrase (e.g., \textit{a dog that is brown; a dog that is
  blue}). These two variables were crossed which yielded four
experimental conditions. For example, one story was: “Sally loves
animals. She decided to get a pet. So she went to the pet store to see
what was there.  Sally immediately set her eyes on a [\textit{blue
  dog/brown dog/dog that was blue/dog that was brown}].  She picked
him up and knew instantly that he was going to be a great companion
for her.” Only one of the expressions within the square brackets was
presented to a particular participant. The items were counter-balanced
such that each participant saw an equal number of stories in each of
the four conditions and each item was seen only once by each
participant. Order of presentation was randomized for each
participant. The full list of target items (i.e., the unusual, normal,
and head noun) is listed in the Appendix.

Participants viewed the stories one at a time on a computer
screen. They were instructed to read each passage carefully and were
allowed as much time as necessary to complete the task.  After each
story, participants answered two questions about the story. The first
question required people to recall the referent of the target noun
phrase from the story. It specifically required the participant to
respond by describing the target concept. For example, a question
might ask \textit{What kind of pet did Sally get?} The participant
typed in their answer. The second question was also associated with
the passage, and asked about another aspect of the story.

\subsection{Results}

Two of the authors classified the responses into four categories based
on how the participants referred to the antecedent: modifier-noun
phrase (e.g., \textit{blue dog} or \textit{brown dog}), full phrase
(\textit{dog that is blue} or \textit{dog that is brown}), and head
noun only (\textit{dog}). In addition, a fourth category was used for
“other” responses. Three main types of responses fell under this
category.  The first were responses that did not provide a specific
answer (e.g., “I don’t know”, “it doesn’t say”). The second were
responses that did not address the question (e.g.,
% “What kind of pet did Sally get” was responded to with “pet”;
“What does Nathan cut quickly” was intended to elicit either green or
yellow grass, but the participant responded “because his parents are
coming home”).
% “What is the only thing that bothers Amada about the place they go?”
% was intended to elicit “smokeless/smoky cigarettes” but the
% participant responded “hey go to parties”; “What fruit does Anna
% bake her pie with?” was responded with “fruit” or “fruit from the
% market in the contry”; “Where is Lucy going to start her education”
% was intended to elicit either clown school or public school, but the
% participant responded “clown”).
The third type of response did not directly refer to the target
reference (e.g.,
% “What kinds of songs are presented on the demo?” was intended to
% elicit either songs with explicit or meaningful lyrics, but the
% participant responded “only ones written by the lead singer”; “Where
% are the different types of trees located” was intended to target
% either candy forests or boreal forests, but some participants
% responded “they are located in Canada”; “What is the house on the
% corner giving out to the children” was intended to elicit “carrot
% candy or sweet candy” but the participant responded “carrots”;
“What does Katie wear to keep her feet warm” was intended to elicit
snake slippers or soft slippers but the participant responded “fuzzy
slippers”).
% “Who will Sandy bring home for her parents to meet” was responded
% with “Sandy will bring home the nice people she meets in class”
% rather than monster/school friends.)

Inter-rater agreement was 100\%. Table~\ref{tab:no-responses} displays
the number of responses (for each condition) in each
category. Overall, participants generally did include modifying
information; modifying information was provided in 962 out of 1456
responses, and the vast majority (84\%) of these responses were in the
form of a modifier-noun phrase. The responses that were coded as Other
were not included in further analyses and, thus, the percentage with
which a category was used within each of the four experimental
conditions was calculated based only on responses in the form of a
modifier-noun phrase, full phrase, and head noun only.

\begin{table}
  \caption{Number of responses and row percentages (in parentheses)
    for each condition that were modifier-noun phrase, full phrase,
    head noun only, or other. Each row sums to 364.\todo[inline]{Can we use an abbreviation for ``modifier-noun phrase'' (e.g. mod-n phrase) in the table heading?  --- Yes, that will be OK.  either m-n phrase or mod-n phrase is fine}}
  \label{tab:no-responses}
  \resizebox{\textwidth}{!}{\begin{tabular}{ll*{4}{r@{ }S[table-format=2.2,tight-spacing,table-space-text-post=),table-space-text-pre=(]}} \lsptoprule
    \multicolumn{2}{c}{Experimental Condition} & \multicolumn{8}{c}{Response Type} \\\cmidrule(lr){1-2}\cmidrule(lr){3-10}
    Property & Form & \multicolumn{2}{c}{modifier-noun phrase} & \multicolumn{2}{c}{full phrase} & \multicolumn{2}{c}{noun} & \multicolumn{2}{c}{other} \\ \midrule
    non-distintive  & modifier-noun phrase & 220 & (60.44) & 5 & (1.37) & 91 & (25.00) & 48 & (13.19) \\
    non-distintive  & full phrase          & 111 & (30.49) & 48 & (13.19) & 157 & (43.13) & 48 & (13.19) \\
    distintive      & modifier-noun phrase & 303 & (83.24) & 2 & (0.55) & 37 & (10.16)  & 22 & (6.04) \\
    distintive      & full phrase          & 177 & (48.63)& 96 & (26.37)& 59 & (16.21) & 32 & (8.79) \\\midrule
                    & Total                & 811 & (55.70) & 151 & (10.37) & 344 & (23.63) & 150 & (10.30) \\
    \lspbottomrule
  \end{tabular}}
\end{table}

We conducted two separate analyses. The first analysis focused on
whether Form and Distinctiveness affected the likelihood of including
modifying information. The second analysis examined whether Form and
Distinctiveness influenced the form (e.g., full phrase vs. modifier-
noun phrase) in which the modifying information was conveyed. In both
analyses, the dependent variable was binary (i.e., is modified vs. not
modified for the first analysis, and compound vs.  phrase for the
second analysis) and, consequently, we used the \textit{melogit}
function in Stata 15 to fit a mixed-effects model for binary
responses. The experimental variables, Form and Distinctiveness, were
included as fixed effects, and subjects and items were included as
crossed random effects. The estimates of the fixed effects are
reported as log odds.

To examine whether the syntactic form (e.g., \textit{wilted flowers}
vs. \textit{flowers that are wilted}) in which the information had
been presented in the story and the distinctiveness of the property
influenced the likelihood of including modifying information when
referring to the antecedent, we fit a model in which the dependent
variable was whether the participant’s response included modifying
information; modifier-noun phrase and full phrase responses were coded
as 1 and the head noun only responses were coded as 0. Both the
distinctiveness of the property and the form in which the information
was presented in the story influenced whether modifying information
was included in the response. Participants were more likely to provide
modifying information when the property presented in the story was
distinctive (e.g., \textit{wilted} as a property of \textit{flowers})
rather than usual (e.g., \textit{fresh} as a property of
\textit{flowers}), 86\% vs. 61\%, $b=1.48, \SE=0.24, z=6.22, p<0.0001$, and
when the property had been presented as a modifier-noun phrase rather
than a full-phrase (81\% vs. 67\%), $b=-1.15, \SE=0.19, z=-5.96,
p<0.0001$. The two predictor variables (Form and Distinctiveness) did
not interact with each other, $b=0.46, \SE=0.31, z=1.48, p=0.14$.

The second analysis was conducted using only the responses that
included modifying information (i.e., only the full phrase and
modifier-noun responses) so that we could test whether the form in
which the modifying information was presented in the story and the
distinctiveness of the property influenced the way in which
participants conveyed the modifying information in their response. The
dependent variable corresponded to whether the response was a
modifier-noun phrase (1 = modifier-noun phrase and 0 = full
phrase). Participants were more likely to provide a modifier-noun
response when the story used a modifier-noun form (predicted $M=0.99,
\SE=0.009s$) than when the story used a full phrase form ($M=0.64, \SE=0.05$),
$\chi^2(1)=31.47, p<0.0001$. Note that because there are only two
levels of the variable, the reverse is also true: namely, that
participants are more likely to provide a full-phrase response when
the story used a full-phrase form than when the story used a
modifier-noun form. The type of property used in the story did not
strongly influence whether participants used a modifier-noun form,
$\chi^2(1)=3.05, p<0.08$.

Distinctiveness and Form interacted, $b=-1.99, \SE=0.58, z=-3.41, p=0.001$,
and, therefore, we examined the simple effects at each level of
form. Distinctiveness of the property had no effect on whether the
response was a full phrase or modifier-noun phrase when the modifying
information was presented as a full phrase, $\chi^2(1)=2.43,
p<0.12$. However, when the modifying information was presented as a
modifier-noun phrase, the response was more likely to be a
modifier-noun phrase when the property was unusual/distinctive than
when the property was normal, $\chi^2(1)=8.91, p<0.003$.


\section{Discussion}

We explored two aspects of the production of multiword referential
expressions: inclusion of modifying information and syntactic form,
with a particular focus on modifier-noun phrases (e.g., \textit{blue
  dog} and \textit{brown dog}) and full noun phrases (e.g.,
\textit{dog that is blue} and \textit{dog that is brown}). The
experiment directly pitted the bias towards truncation against the
bias towards re-using syntactic expressions. The findings make three
primary contributions to the literature on multiword
expressions. First, we demonstrate the influence of
semantic/conceptual knowledge on the inclusion of modifying
information. In particular, the degree of conceptual contrast seems to
be critical in determining whether modifying information is included
when the referential expression is produced. Second, our results
reveal the primacy of modifier-noun phrase constructions (over full
phrase constructions) as a means of conveying that information. Third,
while it is possible that there are small effects of syntactic
repetition, or a general bias to use shorter syntactic forms for a
reference to an already identified object from the story, the bias
towards the modifier-noun phrase appears to be the main driver of the
syntactic form of the referential expression, at least in this
particular communicative task.

\subsection{Including modifying information}

Previous research using visual displays of objects found that
over-specification was more likely when a property was visually
distinctive or salient such as when one object was a different color
than other objects in the display (e.g., in a visual display in which
one dog is blue and the others are orange). The current results extend
this finding to a situation where the objects are not physically
present and the distinctiveness of a property is based on conceptual
knowledge about the modifier and head noun concepts. For example,
\textit{blue} is distinctive for \textit{dogs} but not for
\textit{skies}.  The knowledge needed to determine distinctiveness
comes from past history and knowledge of the concepts involved rather
than from visual information that is presented in the experiment.
Therefore, our finding suggests that people are sensitive to
conceptual distinctiveness in addition to (as shown in previous
research) referential distinctiveness. To illustrate, in general
language usage, a category name (e.g., dog) typically refers to a
generic type (i.e., to the category of dogs).  However, in our study,
the referent was always a particular category member, not a generic
category. Whether participants used a generic label or modified
construction depended on the distinctiveness of the property (relative
to the head noun category) used in the story. In this respect, our
data highlights the role of a particular type of implicit information,
namely knowledge about the nature of the category-subcategory
similarity. In particular, the category label (i.e., dog) was used
when the particular referent in the story was not unusual; that is,
when the entity being described was similar to the generic
representative of the category. Note that the modifying information
was not required to uniquely identify the referent (i.e., there was
only one dog in the story), yet participants often opted to include
this information, especially when it was distinctive. Thus, the
inclusion of modifying information corresponded to a conceptual
distinction rather than a purely referential one in that participants
were sensitive to semantic and conceptual knowledge about the category
to which the referent belonged.

There are several possibilities for why participants tended to provide
over-specified expressions especially when the referent had a
distinctive property than when it had a normal property. One
possibility is that the distinctive properties are just much more
salient. For example, work on memory has suggested that features that
violate expectations are often noticed and remembered particularly
well (e.g., a skull in an office setting, see \citealt{brewer1981role}).
In general, people make note of properties that are not similar to
those they have seen before and, when communicating, they might prefer
to explain these differences to others in the simplest way possible \citep{garrod1987saying,markman1997creation}. In the
current experiment, the distinctive properties might have been more
noticeable than normal properties, and this difference might have
prompted participants to include them in their response. Another
possible explanation is that the distinctive features are more likely
to be incorporated into the representation of the target referent
because they tend not to be true of the head noun. This explanation is
consistent with previous research on novel combined concepts that
suggests that features that are true of the entire phrase but not of
the head noun in general (e.g., \textit{white} for \textit{peeled
  apples}) are more available than features that are true of the head
noun (e.g., \textit{round}) \citep{springer1992feature,gagne1996influence} and also with evidence suggesting that people strongly expect
property differences between things named with modified and unmodified
nouns \citep{gagne2011inferential,gagne2014subcategorisation,spalding2015property}. In our experiment, the normal properties were ones that
tended to be true of the head noun concept, whereas the distinctive
properties were not generally true of the head noun concept. Thus, it
is possible that the distinctive property was more likely to be
integrated into the representation of the target referent than was the
normal property. If so, then distinctive properties would be more
likely to be included in the participant's response than would normal
properties.

\subsection{Selection of syntactic form}

There is some tendency to reproduce the syntactic form in which the
information was first presented; Responses using a modifier-noun
phrase are common, but are even more used when the story also uses a
modifier-noun phrase then when the story uses a full-phrase.
Furthermore, although responses using a full response were relatively
rare, the vast majority of responses that used a full phrase (n=144)
were produced when the story also used a full phrase whereas only 7
responses using a full phrase were produced when the story did not use
a full phrase. This finding is consistent with previous research on
syntactic priming \citep{bock1986syntactic,bock1990framing} that found that
people are more likely to produce passive constructions when
describing a scene when previous sentences contained passive
constructions than when previous sentences did not contain passive
constructions. The current experiment examined part of a sentence,
namely, the structure of a noun phrase, and also found support for
syntactic priming.

However, the selection of syntactic form was not completely determined
by the form presented in the story. Instead, there was a strong
preference toward using a modifier-noun phrase (e.g., wilted flowers)
rather than a full noun phrase (e.g., flowers that are wilted).
Previous work on referential communication has indeed shown an overall
trend towards the use of shortened expressions
\citep{brennan1996conceptual,markman1997creation} and analyses of text
corpora also show evidence of text compression
\citep{marsh1984computational}. Thus, the preference for a
modifier-noun phrase might reflect a tendency to select a
syntactically simpler construction. Modifier-noun phrases are
syntactically simpler than full noun phrases and yet still provide
information that allows the reader/listener to identify a subcategory
of head noun (e.g., \textit{ceramic cup} refers to a particular
subcategory of the category \textit{cup}). Thus, modifier-noun phrase
constructions offer a balance between syntactic simplicity and
informativeness. At the same time, there was little evidence to
suggest that participants selected a head noun only structure over a
modifier-noun structure, even though head noun only structures are
syntactically simpler than modifier-noun phrase structures. That is,
rather than exhibiting an overall bias towards shortening, per se, our
data indicate a bias towards modifier-noun phrase use, which suggests
that modifier-noun phrase might have a special status in the
language. Although full phrases (e.g., \textit{flowers that are
  fresh}) were almost always shortened (to either a modifier-noun
phrase or noun, e.g., \textit{fresh flowers} or \textit{flowers}),
modifier-noun phrases were rarely shortened to noun-only.  Thus, the
use of a modifier-noun phrase rather than a full phrase might reflect
something about the special status of modifier-noun phrases rather
than a general bias toward syntactically simple constructions, per
se. That is, it seems likely that modifier-noun phrases are
particularly useful for conveying subcategory information. People are
sensitive to overt cues that indicate the existence of a contrast set,
such as the presence of the word \textit{only}, and the inclusion of
this cue affects the relative ease of resolving main clause/reduced
relative clause ambiguities \citep{sedivy2002invoking}. Perhaps the
inclusion of modifying information in the context implied the
existence of a contrast set. This might have encouraged people to use
a modifier-noun phrase when referring to the target referent because
this construction indicates a contrast set
\citep{markman1991categorization}.

In sum, we see some evidence for syntactic priming in that the form of
the presentation in the story could reduce the bias to producing
modifier-noun phrases, but the influence of the prior form was
relatively weak in that it was not able to overturn the strong
preference for modifier-noun phrases constructions. Similarly,
although we see some degree of shortening of the referring phrase,
there still seems to be a preference for maintaining at least a
modifier-noun construction, rather than just a generic noun, even
though no modifying information was required in order to identify the
referent in the story. This was particularly true when the modifying
information was atypical.


\section{Conclusion}

Our data reveal that the context in which the linguistic expressions
are used provides useful cues as to the form that the linguistic
expression will take and provide insight into the expectations/biases
that languages users use during referential communication. During
conversation and referential communication, modifier-noun phrases
(e.g., \textit{rotten apple}) are produced for several reasons
including distinguishing among potential referents and maintaining
conversational pacts. The current experiment demonstrates that
modifier-noun phrases also are produced in order to highlight
conceptually distinctive properties. The finding that distinctiveness
influenced the use of modifying information provides insight into how
people use multiword expressions to convey information about how they
are conceptualizing the various entities about which they are
communicating. In particular, the form of the linguistic construction
(e.g., noun versus modifier-noun phrase) provides useful cues as to
the intended meaning. Furthermore, although the participants were
somewhat sensitive to the syntactic form with which the target was
presented, there was a strong bias for the modifier-noun phrase form.
In sum, it appears that modifier-noun phrases have a privileged status
among multiword expressions and provide a good compromise between
competing principles of conveying sufficient information and using
simple syntactic structures.


% Bibliography:

%\clearpage
{\sloppy
  \printbibliography[heading=subbibliography,notkeyword=this]
}

\clearpage
\section*{Appendix}

\begin{table}
  \caption{Full list of target items showing the unusual properties,
    normal properties, and the head noun.}
  \label{tab:}

  \begin{tabular}{lll}
    \lsptoprule
    \multicolumn{2}{l}{Properties} & Head noun \\ \midrule
    
    blue & brown & dog\\
    fresh & wilted & flowers\\
    rotten & red & apples\\
    soggy & crisp & crackers\\
    polluted & blue & lake\\
    curdled & white & milk\\
    rubbery & savory & chicken\\
    soap & shoe & shop\\
    melted & frozen & ice cream\\
    blurry & glossy & photographs\\
    explicit & meaningful & lyrics\\
    green & orange & fire\\
    stale & soft & buns\\
    yellow & green & grass\\
    clown & public & school\\
    carrot & sweet & candy\\
    snake & soft & slippers\\
    cold & hot & shower\\
    sour & sweet & honey\\
    monster & school & friends\\
    crashing & flying & planes\\
    candy & boreal & forests\\
    gravy & train & station\\
    plastic & coloured & chalk\\
    purple & morning & sunshine\\
    plaid & school & pants\\
    smokeless & smoky & cigarettes\\
    pickle & juice & pitcher\\
    \lspbottomrule
  \end{tabular}
\end{table}


\end{document}
